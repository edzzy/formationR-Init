PREMIERS PAS AVEC R ET RSTUDIO
\begin{document}
\section{L'environement RStudio}

\paragraphe{Objectif : } Repérer les différentes fenêtres de RStudio, Savoir changer de répertoire de travail.
Ouvrir le logiciel RStudio.

Présentation des différentes fenêtres de RStudio et personnalisation.

1	EXERCICE : CHANGER DE REPERTOIRE DE TRAVAIL

Fenêtre en bas à droite  ‘FILES' bouton  -> ‘…' choisir Bureau->Formation->J1->TutoR
                           bouton -> ‘More' choisir  ‘Set as Working Directory'

\section{Utilisation de R comme calculatrice avec objet simple}
Objectif : comprendre le concept de variable, stockage d'une valeur dans une variable, calcul simple, variable prédéfinie. Connaitre quelques symboles particuliers. Utiliser l'auto-complétion et l'historique.
1	EXERCICE : CALCUL SIMPLE AVEC LES VARIABLES
Le symbole > en début de ligne (chevron), permet d'indiquer une ligne de commande : il n'est pas à recopier.
Le symbole \# permet d'indiquer que la ligne est un commentaire. Cette ligne est là comme information pour l'utilisateur et n'est pas interprétée par R.
Tester dans la console ces différents exemples :
\#opérateurs :    + - * /
> 50+50
\#stockage d'une valeur dans une variable
> nombreX = 50.8
\#Accès à la valeur stockée dans la variable nombreX
\#Utiliser l'auto-complétion. Commencer par écrire nomb puis appuyer sur la \#touche de tabulation.
> nombreX

\#Calcul avec la variable 				
> nombrex + 25
Pourquoi l'opération ne marche-t-elle pas ? Corriger et exécuter la ligne de commande.

------------------------------------------------------------

\#Utilisation de plusieurs variables
\#stockage de la variable nombreY (le symbole <- est identique à =)
> nombreY <- 7.4
\#Utiliser l'auto-complétion, si plusieurs solutions existent vous pouvez  \#continuer de taper le nom de la variable pour que le choix soit plus \#restreint puis utiliser les flèches du clavier pour choisir la bonne \#variable. Une fois sélectionnée, appuyer soit sur entrée.
> nombreX + nombreY
\#Stockage du résultat de l'opération dans la variable sommeXY
\#Conseil : utiliser l'historique en appuyant sur la ‘flèche haut'
\#on peut retrouver des commandes déjà écrites.		
> sommeXY = nombreX + nombreY		
\#Opération avec variable et constante déjà existante.
> sommeXY					
> rayon = nombreX
> rayon
> pi
> A = pi * rayon^2 \#air d'un cercle en utilisant la variable pi déjà défini dans R
\#D'autres constantes existent : letters, LETTERS, month.name…


\section{Vecteurs}
Un vecteur est un objet qui contient plusieurs éléments.
Objectif : Savoir créer un vecteur,  comprendre le comportement d'une opération de vecteur. Accéder à une valeur particulière

1	EXERCICE : OPERATIONS SUR LES VECTEURS
\#création d'un vecteur numérique
> monVecteur1 = c(20,45,78,12)	
Quelle est la longueur (nombre d'indice) de ce vecteur ? 

------------------------------------------
\#Accès à l'élément d'indice 1
> monVecteur1[1]				
Accéder aux élément d'indice  3 , 4 puis 5.
Que se passe-t-il lorsque que l'on dépasse la longueur du vecteur ?

-----------------------------------------------
\#Accès aux éléments d'indice de 1 à 3
> monVecteur1[1:3]			
\#opération sur le vecteur 
> monVecteur1 + 5.5			
Quel est le comportement de l'addition d'un nombre sur un vecteur ?

------------------------------------------------
\#addition de deux vecteurs de même taille
> monVecteur2 = c(10,100,5,2)
> monVecteur1 + monVecteur2		
Quel est le comportement de l'addition de deux vecteurs entre eux ?

-----------------------------------------------
\#addition de deux vecteurs de taille différente :
> monVecteur3 = c(1,2,3) 
> monVecteur1 + monVecteur3
       
> monVecteur4 = c(50.2,45.5)
> monVecteur4 + monVecteur1
Addition deux vecteurs de taille différentes et quelle est le comportement de l'opération?

--------------------------------------------------
\#Création d'un vecteur avec chaine de caractère.
> monVecteurA = c("A","B","C")
> monVecteurB = c(A,B,C)
Pour quelle raison  monVecteurB n'a pas pu être créer ?


---------------------------------------------------



\section{Utiliser des fonctions}
Objectif : utiliser la documentation d'une fonction, exécuter une fonction avec un ou plusieurs argument(s).
Les fonctions se terminent par des parenthèses. Elles retournent un résultat qui peut être varié (un objet modifié, une information…)
Le résultat de la fonction peut être stocké dans un objet pour être utilisé par la suite.
A l'intérieur des parenthèses, la fonction peut comporter des arguments séparés par des virgules.
Une fonction à déjà été utilisé plusieurs fois dans l'exercice précédent. Laquelle ?

--------------------------------------------------------
Utilisation de la documentation :
Il existe une page de documentation pour la majorité des fonctions, qui est disponible en écrivant :
\# deux façon Accé à l'aide pour une méthode
\# ?nomDeLaFonction 
  \#  help(nomDeLaFonction) 
 Ou auto-complétion (touche tab) puis une fois la fonction sélectionnée la touche F1.

La fenêtre History permet de retracer l'ensemble des commandes exécutée.  Lors du changement de Work directory, une fonction a été exécutée.  Retrouvez cette fonction. Puis ouvrir sa page de documentation.
A quoi correspond l'argument de cette fonction ?

---------------------------------------------------------------

1	EXERCICE : FONCTIONS SANS ARGUMENT 
\#Exécuter les commande suivantes :
> getwd() 	
> ls() 	
> dir() 	

A quoi sert chacune de ces 3 fonctions? A quelle fenêtre de Rstudio la fonction ls() correspond ? Et la fonction dir() ?


------------------------------------------------

2	EXERCICE : FONCTIONS AVEC ARGUMENTS
\#Calcul de la median de monVecteur1
> median(x=monVecteur1)
\#equivalent à
> median(monVecteur1)

\#Somme de tous les éléments de monVecteur1
> sum(monVecteur1)
\#Somme de chaque élément des vecteurs.
> sum(monVecteur1,monVecteur2,monVecteur3, monVecteur4)
Certains arguments possèdent des valeurs par défaut. Ils n'ont pas nécessairement besoin d'être modifiés.
Regarder la documentation de median ou sum pour trouver un tel argument.
\#importation de données issues d'un fichier
L'auto-complétion est aussi très utile pour connaitre les arguments d'une fonction, une fois la première parenthèse ouverte,  appuyer sur la touche tab, la liste des arguments apparait et appuyer sur entrée pour valider l'argument sélectionné. Répéter pour chaque argument.
> maTable<-read.table(file="Rmatrice.txt",header=TRUE,row.names=1,sep="\t")
\#équivalent à 
> maTable<-read.table("Rmatrice.txt",header=TRUE,row.names=1,sep="\t")
Le premier argument est souvent la variable ou le fichier qui sera traité c'est pourquoi l'on omet souvent le nom de l'argument (souvent file pour un fichier x pour une variable).
\section{Matrice}
Objectif : importer des données issues d'un fichier, sauvegarder des objets dans un fichier, calcul sur les matrices.
1	EXERCICE SUR LES MATRICES
Les matrices (matrix) sont des objets assimilés à des tableaux, leur contenu est uniquement numérique.
La fonction read.delim est identique à read.table mais certains arguments n'ont pas les mêmes valeurs par défaut. (Si vous regardez la documentation de read.delim vous remarquerez qu'il s'agit de la même que read.table)

\#Importation de données
> maMatrice0=read.delim("Rmatrice.txt")

\#Obtenir de l'information
> dim(maMatrice0) \#dimension de la matrice
> head(maMatrice0) \#premiers éléments de la matrice (lignes)

Quelle est le nombre de lignes et de colonne de la matrice ?

-----------------------------------------------
> maMatrice=read.delim("Rmatrice.txt",row.names=1)
> head(maMatrice)
Que signifie l'argument row.names ? :

---------------------------------------------------
Effacer un objet
> rm(maMatrice0)
\#Propriété de la matrice :
> colnames(maMatrice) \#nom des colonnes
> maMatrice[,1] \# première colonne
> maMatrice$NT_1 \# éléments de la colonne ayant pour nom NT_1
> head(maMatrice[,1]) \#premiers éléments de la 1ere colonne
> ech1= maMatrice[,1]
> echNT1=maMatrice$NT_1 \#premiers éléments de la colonne ayant pour nom NT_1

\#Calcul avec les matrices
> maMatriceLog=log(maMatrice) \#Log tous les éléments de la matrice
> head(maMatriceLog)
Fonction apply permet de travailler sur les colonnes ou sur les lignes.
\#Calcul uniquement sur les colonnes avec lafonction apply
> geneMedian=apply(maMatrice,2,median)
\#le deuxième argument ‘2' indique que le calcul se fait sur les colonnes. 
\#Le troisième argument ‘median' signifie que l'on applique (apply) le calcul de la médiane sur toutes colonnes.
> geneMedian
> length(geneMedian)
Question : De la même façon créer un profil médian (échantillon médian).

\#A compléter
> profilMedian = apply(

\#Sauvegarde dans un fichier
> write.table(x=maMatriceLog, file="logmat.txt", col.names=NA ,
row.names=TRUE, quote=FALSE, sep="\t")


\section{Graphiques}
Objectifs : savoir utiliser des fonctions graphiques simples, savoir sauvegarder une image.

1	TRACER DES GRAPHIQUES
\#Utilisation basique
> boxplot(maMatriceLog)
\#Avec quelques paramètres
> boxplot(maMatriceLog, col="blue" , las="2")

\#Sauvegarde dans un fichier image
Dans l'onglet Plots : Export-> ‘Save Plot As Image'
File name :  boxPlot
La même chose avec la commande : 
> dev.print(jpeg,file="boxPlot.jpeg")

La fonction plot permet de tracer une série de données contre une autre (x,y).
Par exemple on sélectionne la première colonne.
> ech1 = maMatrice[ ,1]

En supposant que le profil median calculé précédemment se nomme profilMedian :
\#Graphique simple (l'argument log permet d'obtenir une échelle en log pour les axes mentionnés)
> plot(x = profilMedian, y = ech1, log="xy")

\#Le même graphique avec des paramètres supplémentaires 
> plot(x = profilMedian, y = ech1,log="xy" ,pch=19,col="darkgreen",xlab="Profil Median", ylab="Echantillon 1", main="Profil median VS Echantillon 1")

\section{Jeu des 5 erreurs : le débuggage}
Objectif : Corriger des erreurs dans une ligne de commande. Se faciliter la vie en utilisant l'historique des commandes et l'auto-complétion.
1	DEBUGGAGE
Ouvrir le fichier bugTuto.R
Corriger la ligne de commande afin de ne plus avoir de message d'erreur.
Méthode : Exécutez la ligne de commande telle quelle  puis corriger l'erreur annoncée, relancer la ligne de commande et ainsi de suite jusqu'à qu'il n'y ait plus de message d'erreur. Pour chaque erreur trouvée, écrire la cause de l'erreur dans le fichier bugTuto.R.
Utilisation de l'historique : dans la console, appuyez sur la flèche du haut pour remonter dans l'historique et sur celle du bas pour en redescendre.

Une fois la commande corrigée, cliquez sur l'onglet ‘History' à coté de ‘Workspace'. Sélectionner la commande corrigée (la dernière ligne) et cliquer sur le bouton ‘To Source' (raccourci clavier : Maj+Enter).
Sauvegarder votre fichier.
\section{Executer un script}
1	DIFFERENTES METHODES POUR EXECUTER UN SCRIPT 
Ouvrir le fichier ‘scriptTuto.R'.
Pas à pas :
Dans la fenêtre source, placer le curseur sur la première ligne :
Cliquer sur l'icone « Run » ou le raccourci clavier Ctrl+Entrée
Par bloc :
	Sélectionner les lignes à exécuter avec la souris ou le clavier Maj+fleches.
	Puis de la même façon que pour le pas à pas cliquer sur « Run » ou Ctrl+Entrée.
Tout le fichier :
	Cliquer sur Source ou le raccourcis clavier Maj+Ctrl+S


\section{Ecrire et utiliser une fonction}
Objectif : connaitre la syntaxe de la création d'une fonction. Savoir mettre en mémoire une fonction pour l'utiliser (‘sourcer').
1	FONCTION SIMPLE 
Ouvrir le fichier fonctionTuto.R et le fichier script2Tuto.R :
Répéter l'exercice précédent avec le fichier script2Tuto.R.
On obtient un message d'erreur :  la fonction addition n'existe pas pour R. Pour lui faire connaitre il faut ‘'sourcer'' le fichier qui contient la fonction.

Ecrire une fonction similaire à addition du style : soustraction, multiplication…
Utiliser la fonction écrite dans le fichier script2Tuto.R .
Exécuter tout le script.

Pour éviter d'oublier de ‘sourcer' le fichier, la méthode standard est d'écrire au début du fichier la commande  :
> source("fonctionTuto.R")

Une autre option propre à RStudio est de cocher la case ‘Source on Save' pour le fichier fonctionTuto.R. Cela a pour effet de ‘sourcer' automatiquement le fichier fonctionTuto.R à chaque sauvegarde de ce fichier.

2	UNE FONCTION PLUS COMPLEXE

Grâce aux exercices précédents, écrire une fonction qui prend en argument une matrice  et qui retourne le log du profil médian (médiane des gènes).















\section{Bibliotheque (library)}
Les bibliothèques sont des ensembles de fonctions et objets particuliers. La bibliothèque limma qui est adaptée à l'analyse de donnée de puce à ADN sera utilisée.
Pour utiliser les fonctions de ces bibliothèques, il faut charger la bibliothèque. Pour cela il suffit d'écrire
> library("limma")
ou alors d'aller sur l'onglet Packages et de cocher la case à coté de limma. Si vous cliquez sur limma, vous aurez accès à toute la documentation de la bibliothèque.

\section{Pour aller plus loin : les facteurs}
Fonctions utilisées : 
 rep(), which(), factor()
Création des facteurs.
> echFacteur <- factor(c(rep("NT",6),rep("CYT6H",8,),rep("CYT48H",6)))

1	EXTRAIRE LES ECHANTILLONS NT

> names(echFacteur) \#nom des éléments de echFacteur
> colnames(maMatrice) \# nom des colonnes de maMatrice
> all( names(echFacteur) == colnames(maMatrice)) \# Répond TRUE si les noms de echFacteur et les noms des colonnes de maMatrice sont identiques. 

Construction de la commande par étape :

> echFacteur == "NT"  \# teste si chaque élément est égal à « NT »
> which(echFacteur == "NT") \# Indices qui répondent à la condition.

Commande finale :

> NT <- maMatrice[,which(echFacteur == "NT")] \#Sélectionne les colonnes de la matrice correspondant aux indices ci-dessus.

De la même façon, récupérer les échantillons CYT6H puis CYT48H




\end{document}

\documentclass[a4paper]{article}

\usepackage[utf8]{inputenc}
\usepackage[T1]{fontenc}
\usepackage[francais]{babel}
\usepackage[top=2cm, bottom=2cm, left=2cm, right=2cm]{geometry}
\usepackage{graphicx}
\usepackage{float}
\usepackage{lmodern}
\usepackage{textcomp}
\usepackage{underscore}
\usepackage{longtable}
\usepackage{hyperref}
\usepackage{fancyhdr}
\usepackage{changepage}


\renewcommand{\labelitemi}{$\bullet$}
\renewcommand{\labelitemii}{$\bullet$}

\title{\input{title}}
\author{\input{author}}

\lhead{}
%\lfoot{\emph{Rapport d'analyse P118}}
\begin{document}
\makeatletter
  \begin{titlepage}
	    \centering
%		      {\large \textsc{Plateforme BiRD}}
			\hfill
			  \includegraphics[width=0.2\textwidth]{logo1.jpeg}\\
			  \vspace{3em}
			  % \includegraphics[width=0.35\textwidth]{logo2.jpeg}\\
			   \includegraphics[width=1\textwidth]{bandeau.jpeg}\\
			  \vspace{1cm}
			    {\large\textbf{Rapport d'analyse}}\\
			  \vspace{1em}
%			   \vfill
			 {\LARGE \textbf{\@title}} \\
			     \vspace{2em}
			{\large \@author} 
			   \vfill
			   {\large\textbf{	\@date}}\\
	\end{titlepage}
\makeatother
%\maketitle
\tableofcontents
\newpage
\pagestyle{fancy}
\section{Questions posées}
\input{resume.tex}
\section{Description de l'analyse}
L'analyse comporte deux volets :
\begin{enumerate}
	\item Le premier est de fournir des données nettoyées et analysables. 
	\\ Cette partie comporte : 
	\begin{itemize}
		\item le contrôle qualité des échantillons, 
		\item la normalisation, 
		\item le filtrage des données 
		\item la visualisation des données par clustering hiérarchique. 
	\end{itemize}
	\item Le second est un guide pour l'interprétation biologique grâce à deux méthodes d'analyses :
	\begin{itemize}
		\item une analyse globale : recherche de clusters. 
		\item une analyse par gènes : analyse différentielle.
	\end{itemize}
		Dans les deux cas : annotation fonctionnelle basée sur Gene Ontology.
\end{enumerate}

 Les résultats de ces analyses sont sous la forme de listes de gènes associées à des fichiers d'annotations fonctionnelles. 
\\
L'ensemble de la démarche est décrite \bsc{Figure : }\ref{pipeline}.
\input{pipeline}


\section{Importation des données}

\input{importData}

%\subsection{Matrice obtenue}
%Exemple matrice d'expression brute.(graph)
%\input{graphImport}


\section{Normalisation}
\input{normalisation}

\section{Contrôle qualité des échantillons}
\subsection{Distribution du signal : boxplot}
\begin{itemize}
\item Elles sont réalisées avant et après la normalisation
\item Pour les puces hybridées avec deux fluorochromes, on s'attend à une différence d'intensité entre les deux avant normalisation.
\item Après normalisation cet effet doit être corrigé.
\end{itemize}


\textbf{Résultats :} \bsc{Figure}~\ref{statDes}.
\input{graphStatDesc}
\subsection{Corrélation}
On calcule la corrélation de Pearson de chacun des échantillons par rapport à un Profil médian.
\par
Le profil médian est calculé à partir de la matrice de données d'expression. 
\par
Pour chaque sonde on calcule son signal médian (la médiane)
%\input{exempleProfilMedian}
\par
De manière générale la corrélation varie entre -1(anti corrélé) et 1 (corrélé).
Si elle est égale à 0 il n'y a pas de corrélation.
\par
Nous considérons qu'une corrélation >= 0.8 est une bonne corrélation.
\\
\textbf{Résultats :} \bsc{Figure}~\ref{Correlation}.
\input{graphCorrelation}

\section{Filtrage}
\subsection{Principe}

Le filtrage va permettre d'éliminer les sondes qui ne sont pas informatives pour l'analyse, par exemple les sondes trop faiblement exprimées ou les sondes trop peu variables.
\par
Pour cela, différents critères sont utilisés. Les plus communs sont le bruit de fond et la variabilité (CV ou écart-type)
\subsection{Paramétres utilisés dans l'analyse}

\input{filtrage}

\section{Analyse par Clustering}
\subsection{Définition}
Le clustering hiérarchique est une méthode de classification non suppervisée permettant de construire des groupes (clusters) d'objets similaires à partir d'un ensemble hétérogène d'objets.
\par
\vspace{1em}
Le logiciel utilisé est cluster \cite{cluster} . Les paramétres utilisés pour le calcul sont : 
\begin{itemize} 
	\item la distance :\emph{``Uncentered correlation``} 
	\item le regroupement \emph{``average-linkage``}
\end{itemize}
 
\subsection{But}
\begin{itemize}
\item Visualiser les données et determiner des problèmes dans la matrice d'expression.
\item Visualiser la classification des échantillons.
\item Visualiser les signatures (regroupement des gènes).
\item Sélectionner les clusters d'expression différentielle.
\end{itemize}
La matrice est représentée par les sondes en lignes et les échantillons en colonne. Les valeurs par ligne sont centrées sur la médiane. Une couleur rouge montre une sur-expression de l'échantillon et une couleur verte une sous expression.
%\subsection{Filtrage Clustering : signal median}
%\input{filtrageClustering}
\subsection{Clustering général}

\input{generalCluster}
\subsection{Démarche par clustering (groupe de gène)}
A chaque question binaire : 
\begin{itemize}
	\item Un test de Student est effectué par sonde et une Pvalue lui est attribuée (voir section~\ref{sec:gendiff} pour plus de détail)
\item L'image montre la heatmap resultant du clustering hiérarchique associée aux pvalues de la question posée. 
\item Ces pvalues sont transformées en appliquant la formule -log10(pvalue) afin de mieux visualiser les données
\item Des pics distinguent les clusters répondant à la question posée.
\item Afin de lisser la courbe nous calculons une \emph{moyenne mobile}. La moyenne des -log10(pvalue) est faite par pas de 200.
\end{itemize}
%\input{graphExempleCluster}
\subsection{Questions Binaires : détection des clusters }
\input{graphCluster}
%\input{clustBinaire}

\subsection{Annotation Fonctionnelle des clusters}
Le principe est d'associer des annotations fonctionnelles aux clusters, en utilisant la base de données Gene Ontology \cite{go}\\
Pour ce faire nous utilisons le logiciel GOminer \cite{gominer} qui fournit pour chaque liste deux fichiers : 

\par

\IfFileExists{AnnotationCluster.tex}{\input{AnnotationCluster.tex}}

\begin{itemize}
	\item Un fichier \emph{*fdrse.txt} contenant les annotations GO associées à différentes informations
	\item Un fichier \emph{*gce.txt} contenant l'association des annotations et des gènes du clusters.
\end{itemize}
\begin{itemize}
\item En-tête du fichier \emph{*.fdrse.txt} :
\begin{description}
	\item[GoID : ] indentifiant de l'annotation dans la base GO 
	\item[Total : ] nombre de fois que l'annotation est retrouvée sur la matrice filtrée
	\item[Change : ] nombre de fois que l'annotation est retrouvée dans le cluster
	\item[Enrichissement : ] calcul de l'enrichissement de l'annotation
	\item[Pvalue : ] significativité de l'annotation
	\item[Term : ] description de l'annotation
\end{description}

\end{itemize}


\section{Analyse différentielle}
\label{sec:gendiff}
Pour chaque question binaire un test de student est effectué pour chacune des sondes. On teste si les moyennes des deux conditions sont significativement différentes.
\\
Multiplier les tests statistiques augmente la probabilité globale de se tromper (faux positifs). Pour limiter cet effet, nous effectuons une correction des pvalues par la méthode de Benjamini-Hochberg. Une sonde est considérée comme différentiellement exprimé entre 2 conditions si sa p-value est < 0.05.
\\

\subsection{Résultats}

Le tableau ci-dessous représente le nombre de sondes significatives avec correction
\par
\input{genDiff.tex}

%\input{genDiffRaw.tex}

\par
\subsection{Annotation fonctionnelle des gènes différentiels}
\IfFileExists{annotGeneDiff.tex}{\input{annotGeneDiff.tex}}{Pas d'annotation fonctionnelles significatives}

\section{Conclusion de l'analyse}
\input{conclusion}
\newpage
\appendix
\section{Annexes}
\subsection{Outils complémentaire}
\input{gorilla}

\subsection{Info Session R}
\input{info}
\section{Bibliographie}
\bibliography{bibl}{}
\bibliographystyle{apalike}
\newpage
\section{Fichiers rendus}
\input{files}
\end{document}


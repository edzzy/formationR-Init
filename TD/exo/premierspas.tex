\section{Premiers Pas}
\subsection{R est une calculette}

\lstinputlisting{code/calculsimple.R}
\paragraph{EXERCICE} : 
 Calculer la différence entre votre dernière année à l'université
 Et votre année d'obtention du bac et divisé par la différence entre 2013 et votre
 année de naissance. Multiplier par 100 pour obtenir un pourcentage du temps passé 
 à l'université.
\lstinputlisting{code/calculsimple2.R}
\paragraph{EXERCICE} : 
Répéter l'exercice précédant mais en utilisant des variables. Plusieurs étapes peuvent être nécessaires.
\subsection{Créer des objets et les utiliser}
\subsubsection{Vecteurs}
\lstinputlisting{code/vecteur.R}
\paragraph{EXERCICE} En utilisant la fonction \emph{seq()}, Créer un vecteur de nombres impaires de 7 à 77. 
\lstinputlisting{code/vecteur2.R}
\paragraph{EXERCICE} En utilisant les fonctions \emph{rep()} et \emph{paste(}) et \emph{seq()} créer un vecteur identique à \emph{monVecteurA}, mais allant de 1 à 20.
\lstinputlisting{code/vecteur3.R}
\subsubsection{Matrices et data frame}
\lstinputlisting{code/datamatrice.R}

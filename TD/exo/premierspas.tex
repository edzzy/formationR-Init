\section{Premiers Pas}
\subsection{R est une calculette}

\lstinputlisting{code/calculsimple.R}

\paragraph{EXERCICE} : 

 Calculer le temps passé dans votre dernière structure professionelle (université, labo, entreprise \dots)
 puis diviser par la différence entre l'année en cours et votre année de naissance. Multiplier par 100 pour obtenir un pourcentage du temps passé 
 dans cette structure.\par

\lstinputlisting{code/calculsimple2.R}

\paragraph{EXERCICE} : 

Répéter l'exercice précédant mais en utilisant des variables.
\subsection{Créer des objets et les utiliser}

\subsubsection{Vecteurs}

\lstinputlisting{code/vecteur.R}

\paragraph{EXERCICE} En utilisant la fonction \emph{seq()}, créer un vecteur de nombres impaires allant de 7 à 77. 

\lstinputlisting{code/vecteur2.R}

\paragraph{EXERCICE} En utilisant les fonctions \emph{rep()} et \emph{paste(}) et \emph{seq()} créer un vecteur identique à \emph{monVecteurA}, mais allant de 1 à 20. Dans un premier temps utiliser des étapes intermédiares.

\lstinputlisting{code/vecteur3.R}

\paragraph{EXERCICE} Trouver une manière d'accèder au dernier élément d'un vecteur.

\lstinputlisting{code/vecteur4.R}

\subsubsection{Matrices et data frame}

\lstinputlisting{code/datamatrice.R}
\paragraph{EXERCICE} : 
\begin{itemize}
\item Combien de lignes la fonction head() affiche - t'elle par défaut
\item Comment afficher plus ou moins de ligne ? Utiliser la documentation de la fonction head pour trouver.
\item Comment voir la matrice à partir des dernières lignes ?
\end{itemize}
\lstinputlisting{code/datamatrice2.R}
\paragraph{EXERCICE} :
\begin{itemize}
\item Quel type d'objet renvoie la fonction dim()
\item La fonction \emph{ncol()} permet de nous renseigner sur le nombre de colonnes.
Trouver une fonction similaire pour trouver le nombre de lignes
\item Avec la fonction \emph{dim()} afficher uniquement le nombre de lignes
\end{itemize}
\lstinputlisting{code/datamatrice3.R}
\paragraph{EXERCICE} : Recupérer les colonnes 1, 7 et 3 de la matrice \emph{maMatrice}
\lstinputlisting{code/datamatrice4.R}

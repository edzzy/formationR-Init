\section{Écrire et utiliser une fonction}
\paragraph{Objectif }: connaitre la syntaxe de la création d'une fonction. Savoir mettre en mémoire une fonction pour l'utiliser (‘sourcer').
\paragraph{Fonction simple}
Ouvrir le fichier fonctionTuto.R et le fichier script2Tuto.R :
Répéter l'exercice précédent avec le fichier script2Tuto.R.
On obtient un message d'erreur :  la fonction addition n'existe pas pour R. Pour lui faire connaitre il faut 'sourcer' le fichier qui contient la fonction.
\par
Écrire une fonction similaire à addition du style : soustraction, multiplication…
Utiliser la fonction écrite dans le fichier script2Tuto.R .
Exécuter tout le script.
\par
Pour éviter d'oublier de ‘sourcer' le fichier, la méthode standard est d'écrire au début du fichier la commande  :
\begin{verbatim}
> source("fonctionTuto.R")
\end{verbatim}

Une autre option propre à RStudio est de cocher la case ‘Source on Save' pour le fichier fonctionTuto.R. Cela a pour effet de ‘sourcer' automatiquement le fichier fonctionTuto.R à chaque sauvegarde de ce fichier.




















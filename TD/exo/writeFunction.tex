\section{Écrire et utiliser une fonction maison}
\paragraph{Préparation}
Il est plus facile d'écrire des fonctions dans un ou plusieurs fichiers. Créer un nouveau fichier et le nommer \emph{mesFonctions.R}
La syntaxe pour écrire une fonction est la suivante : 
\lstinputlisting{code/synthaxefunction.R}
\paragraph{Ennoncé} : Créer la fonction \emph{ingredientsPateAPizza} qui prend comme argument le nombre de pizza (par défaut 1). Le retour est un vecteur avec les quantités pour X pizzas. : 
\lstinputlisting{code/retour.R}
Les ingrédients pour une pizza
\begin{itemize}
	\item farine : 500
	\item eau : 250
	\item levure : 20
\end{itemize}

\paragraph{Tester sa fonction} : La fonction est écrite dans le fichier fonction.R, mais elle est encore inconnue dans le \emph{workspace}. Pour la mettre en mémoire il faut executer la fonction  \emph{source()}.
\lstinputlisting{code/source.R}
Si il n'y pas d'erreurs de syntaxe, la fonction sera reconnue par R. On le remarque sur la fenêtre Workspace.
\par
On peut maintenant utiliser notre fonction de la même façon que n'importe quelle autre.

\paragraph{EXERCICE} : Créer la fonction \emph{repartIngr} qui prend comme arguments 
\begin{itemize}
\item \emph{vecIngred} Un vecteur de nombre dont chaque indice à le nom d'un ingrédient.
\item \emph{imageName} Une chaine de caractère qui correspond au nom du fichier image.
\item \emph{recipiesName} Une chaine de caractère qui correspond au nom de la recette. 
\end{itemize}

Cette fonction doit créer un graphique avec la fonction \emph{pie} et l'enregistrer sous le même nom que \emph{imageName}

\section{Programmation : les bases}

\subsection{Les boucles}

Le mot \emph{for} permet de répéter une instruction un certain nombre de fois.
\lstinputlisting{code/for.R}
\paragraph{EXERCICE} : Dans un nouveau fichier de votre choix, écrire une boucle \emph{for} exécutant 10 fois  la fonction \emph{ingredientsPateAPizza}

\subsection{Les conditions}
Les instructions \emph{if} et \emph{else} permettent d'executer des parties de codes sous certaines conditions.
\lstinputlisting{code/ifelse.R}
\paragraph{Exercice} : Dans la fonction \emph{ingrédientsPateAPizza} ajouter une instruction \emph{if else} qui vérifie que l'argument nbPizza est bien positif et non nul. Auquel cas, un message du style "Veillez saisir un nombre de pizza positif" devra s'afficher.
 












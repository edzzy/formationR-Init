\section{Écrire et utiliser une fonction et l'utiliser dans un script}
\subsubsection{Préparation}
Il est plus facile d'écrire les fonctions dans des fichiers. Créer un nouveau fichier et le nommer fonction.R
La synthaxe pour écrire une fonction est la suivante : 
\lstinputlisting{code/synthaxefunction.R}
\paragraph{Ennoncé} : Créer la fonction \emph{ingredientsPateAPizza} qui prend comme argument le nombre de pizza, par défaut 1. Elle doit retourner un vecteur indiquant les quantités pour les ingrédient.\par
Les ingrédients pour une pizza
\begin{itemize}
	\item farine : 500
	\item eau : 250
	\item levure : 20
\end{itemize}

\paragraph{Aide} : Pour nommer les éléments d'un vecteur on utilise la fonction names() qui fonctione de la même façon que colnames pour les matrices et data.frame.

\paragraph{Tester sa fonction} : La fonction est écrite dans le fichier fonction.R, mais elle est encore inconnue pour R. Pour la mettre en mémémoire il faut executer la fonction  \emph{source()}.
\lstinputlisting{code/source.R}
Si il n'y pas d'erreurs de syntaxe, la fonction sera reconnue par R ce que l'on remarque sur la fenêtre Workspace.
\par
On peut maintenant utiliser notre fonction de la même façon que n'importe quelles autres.


















\section{Écrire et utiliser une fonction maison}
\subsubsection{Préparation}
Il est plus facile d'écrire des fonctions dans un ou plusieurs fichiers. Créer un nouveau fichier et le nommer mesFonctions.R
La synthaxe pour écrire une fonction est la suivante : 
\lstinputlisting{code/synthaxefunction.R}
\paragraph{Ennoncé} : Créer la fonction \emph{ingredientsPateAPizza} qui prend comme argument le nombre de pizza, par défaut 1. Et retourne un vecteur avec les quantités pour X pizzas. : 
\lstinputlisting{code/retour.R}
Les ingrédients pour une pizza
\begin{itemize}
	\item farine : 500
	\item eau : 250
	\item levure : 20
\end{itemize}


\paragraph{Tester sa fonction} : La fonction est écrite dans le fichier fonction.R, mais elle est encore inconnue dans le \emph{workspace}. Pour la mettre en mémémoire il faut executer la fonction  \emph{source()}.
\lstinputlisting{code/source.R}
Si il n'y pas d'erreurs de syntaxe, la fonction sera reconnue par R ce que l'on remarque sur la fenêtre Workspace.
\par
On peut maintenant utiliser notre fonction de la même façon que n'importe quelles autres.

\paragraph{Exercice} : Créer la fonction \emph{repIngred} qui prend comme arguments 
\begin{itemize}
\item \emph{vecIngred} Un vecteur de nombre dont chaque indice à le nom d'un ingrédients.
\item \emph{imageName} Une chaine de caractaire qui correspond au nom du fichier image.
\item \emph{recipiesName} Une chaine de carataire qui correspond au nom de la recette. 
\end{itemize}


\section{Programation les bases}

\subsubsection{Les boucles}

Le mot \emph{for} permet de répeter une instruction un certain de nombre de fois.
\lstinputlisting{code/for.R}
\paragraph{Exercice} : Dans un nouveau fichier de votre choix. Écrire une boucle \emph{for} exécutant 10 fois  la fonction \emph{ingredientsPateAPizza}

\subsection{Les conditions}
les instructions \emph{if} et \emph{else} permettent d'execter des parties de codes sous certaines conditions.
\lstinputlisting{code/ifelse.R}
\paragraph{Exercice} : Dans la fonction \emph{ingrédientsPateAPizza} ajouter une instruction \emph{if else} qui vérifie que l'arguments nbPizza soit bien positif et non nul. Auquel cas un message du style "Veillez saisir un nombre de pizza positif" devra s'afficher.
 












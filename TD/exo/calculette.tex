\section{Utilisation de R comme calculatrice avec objet simple}
\paragraph{Objectif : } Comprendre le concept de variable, stockage d'une valeur dans une variable, calcul simple, variable prédéfinie. Connaitre quelques symboles particuliers. Utiliser l'auto-complétion et l'historique.
\subsection{Exercice : calcul simple avec les variables}
Le symbole $>$ en début de ligne (chevron), permet d'indiquer une ligne de commande : il n'est pas à recopier. \\
Le symbole \# permet d'indiquer que la ligne est un commentaire. Cette ligne est là comme information pour l'utilisateur et n'est pas interprétée par R.
Tester dans la console ces différents exemples :
\lstinputlisting{code/calculsimple.R}
\lstinputlisting{code/vecteur.R}

\paragraph{Ce qu'il faut retenir}
Comment stocker une variable
L'importance de la casse
utiliser l'auto-completion
Chaine de caractere entre guillement



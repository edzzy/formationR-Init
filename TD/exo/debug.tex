
\section{Jeu des 5 erreurs : le débuggage}

%\lstinputlisting{../../formationR/bugTuto.R}
Le debuggage occupe une part non négligeable du temps pour l'utilateur de R. Avec l'experience, les erreurs se repèrent et se corrigent plus rapidement.
\paragraph{Objectif }: Analyser des messages d'erreurs fréquent et les corriger. 
\paragraph{But} : Corriger la ligne de commande afin de ne plus avoir de message d'erreur.
\paragraph{Méthode} : Ouvrir le fichier \emph{bugTuto.R}. Exécutez la ligne de commande telle quelle  puis corriger l'erreur annoncée, relancer la ligne de commande et ainsi de suite jusqu'à qu'il n'y ait plus de message d'erreur. Pour chaque erreur trouvée, écrire la cause de l'erreur dans le fichier bugTuto.R.
\paragraph{Conseil} 
\begin{itemize}
	\item Utiliser l'historique pour ne pas avoir a réécrire toute la ligne et ajouter de nouvelles erreurs de frappes.
	\item Abuser de l'auto-completion pour éviter les erreurs de frappes ou de syntaxe.
\end{itemize}

